\documentclass[11pt]{article}
%check
% Packages
\usepackage[letterpaper, margin=1in]{geometry}

\usepackage{tabularx}
\usepackage{booktabs}

\title{A longtable + tabularx example, using xltabular}


\newcounter{minuteMeeting}
\newcommand{\mmmm}[0]{Meeting \#\refstepcounter{@ } }

\refstepcounter

\begin{document}
    \mmmm
    \mmmm
    \mmmm
    \mmmm

    \subsection{}

    26-Sep-’22 Yonatan, Michael Development proposal and development details.
    The project will be a therapy application that helps users gather, store and present their information to be used by patients.

    Basic details about the development so far are:
    -    Incremental development was suggested
    -    Phone application is the best way to go currently. A web app might be a choice, but nothing is concrete.
    Suggested Programming languages:
    Dart, Python, Javascript, C++, Javascript
    Stakeholders:
    -    Clients
    -    Therapy doctors and nurses
    Features:
    We talked about custom ratings and descriptions for drugs.


    % Please add the following required packages to your document preamble:
% \usepackage{graphicx}
    \begin{table}[]
        \centering
        \resizebox{\columnwidth}{!}{%
            \begin{tabular}{lll}
                26-Sep-’22 &
                Yonatan, Michael &
                \begin{tabular}[c]{@{}l@{}}

                \end{tabular} \\
                30-Sep-’22 &
                &
                \begin{tabular}[c]{@{}l@{}}
                    Our proposal was accepted and we have three group members.\\      Professor comments are below:\\      1. Patient data requirements with respect to privacy.\\      2. Identifying the stakeholders with who should be able to access data.\\      3. We need to identify the use cases carefully of what we are able to do.\\      4. We have to be able to clearly define the project objective and description.\\      An aggressive timeline has been suggested to keep with coursework. This timeline is subject to change due to other courses requiring our attention.   \\      Github was setup by Yonatan. The report is currently being done.
                \end{tabular} \\
                \begin{tabular}[c]{@{}l@{}}
                    04-Oct-’22\\      05-Oct-’22
                \end{tabular} &
                &
                \begin{tabular}[c]{@{}l@{}}
                    A timeline has been suggested to be finishing up to the user acceptance criteria. \\      QT has been suggested as a platform to build an Android app.\\      \\      The report has been started in LaTex by Yonatan. \\      Proposed statements will be made by the group once we reviewed the work.
                \end{tabular} \\
                14-Oct-’22 &
                Yassine, Yonatan, Michael &
                \begin{tabular}[c]{@{}l@{}}
                    Debate about the code is that we have no firm agreements on project libraries and frameworks.\\      Suggestions include: salesforce as the backend. Flutter or ReactNative has also been suggested. Qt.
                \end{tabular} \\
                \begin{tabular}[c]{@{}l@{}}
                    21-Oct-’22\\      to\\      28-Oct-’22
                \end{tabular} &
                &
                \begin{tabular}[c]{@{}l@{}}
                    A lot of project will be developed on the reading week. \\      Professor wants us to use Docker, and create a Gantt chart. He explicitly states that he wants us to assign backup roles to our task assignment matrix.
                \end{tabular} \\
                01-Nov-’22. &
                Yonatan, Michael &
                \begin{tabular}[c]{@{}l@{}}
                    Currently, our planned architecture will be:\\      - repository style for the databases. Note: scrapped.\\      - Event-driven for reminders Note: scrapped.\\      - Component-based. Note: scrapped.\\      \\      Today's objective:\\      Architecture and design patterns were discussed. \\      Get the backend done behind the scenes. \\      Another meeting: 9:30 am to 11:30 am tomorrow (November 2nd, 2022), want to start writing code.
                \end{tabular} \\
                02-Nov-’22. &
                Yassine, Yonatan, Michael &
                \begin{tabular}[c]{@{}l@{}}
                    We are switching the development process from incremental to a hybrid agile-iterative method. Scope and requirements changed significantly throughout the process. We are also incorporating agile methods to respond to changes over time.\\      \\      Kanban board has been created inside GitHub repository.\\      \\      User stories will be created for UI to make the interface clearer to write components and understand the problems.\\      \\      Drafts have been started but significant changes need to be created. \\      \\      An Android application will be created. The data will be stored locally with some encryption to get\\      us started.\\      \\      Skeleton code wants to be pushed by Yonatan. It will require approval from Michael and Yassine.\\      \\      We want to finish baseline features executed and finished by November 6th.   \\      The list of things we want executed:\\      -    A basic UI - covered by Yassine\\      -    Basic backend - covered by Yonatan\\      -    Database integration - covered by Michael\\      \\      Significant research must be implemented for any implementation to occur.\\      \\      No meeting will be scheduled for tomorrow. A few mini-meetings between Friday to Sunday.
                \end{tabular} \\
                02-Nov-’22. &
                Yonatan, Michael &
                \begin{tabular}[c]{@{}l@{}}
                    Backend talk\\      \\      Significant modifications need to be executed for medication.\\      We will have to split this into 3 main Tables\\      -    med info\\      -    med inventory history\\      -    med intake history\\      \\      Mood will be generated next. \\      \\      Symptoms will be last. \\      \\      Backend sample is expected to be done by Sunday late afternoon.\\      Yassine will present a frontend demo will be presented on November 6th.
                \end{tabular} \\
                14-Nov-’22. &
                Yonatan, Michael, Yassine &
                \begin{tabular}[c]{@{}l@{}}
                    Front end and back end are being implemented. We are integrating today.\\      Our approach has changed. We have recognized that the Model-View-ViewModel   (MVVM) pattern was the best representation.\\      Things to do:\\      -    Integrate front-end and back-end. \\      -    Revise databases.\\      -    Check on scope and extra features. \\      -    Testing and presentation notes to build.
                \end{tabular} \\
                24-Nov-’22. &
                Yonatan, Michael &
                \begin{tabular}[c]{@{}l@{}}
                    Presentation is being worked on. \\      Yonatan changed color scheme and logo.\\       Things to do:\\      1. He wants a better option for moods. A better option would be spinner or radio button. A perfect solution would be adding a slider. \\      2. We need to focus more on accessibility.\\      3. Add statistics.
                \end{tabular} \\
                27-Nov-’22. &
                Yonatan, Michael &
                \begin{tabular}[c]{@{}l@{}}
                    Code must refactored to meet architectural standards. \\      Code will be redone to the exact standard to the specified report\\      \\      Mood will be the focus. Medication and Symptoms are not the primary objective.\\      \\      Deadline to finish the code is tomorrow night. \\      Deadline should be finished by Wednesday night.\\      Deadline to finish the document components is Thursday.
                \end{tabular} \\
                28-Nov-’22. &
                Yonatan, Michael &
                \begin{tabular}[c]{@{}l@{}}
                    Multiple meetings today.\\      We have to go with MVC. MVVM was not feasible given the circumstances.\\      \\      Code is being redone to make the sections better. \\      We want to get the code ready for deployment soon.\\      \\      Things to do:\\      -    Clean up redundant code\\      -    add comments\\      -    fix anything that seems broken.\\      -    Deploy on Docker.\\      -    Another addition of report documents and diagrams.
                \end{tabular}
            \end{tabular}%
        }\label{tab:table}
    \end{table}


\end{document}