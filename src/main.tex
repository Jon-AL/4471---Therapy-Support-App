%! Author = Jonathan
%! Date = 2022-09-26

% Preamble
\documentclass[11pt]{article}
%check
% Packages
\usepackage{mathrsfs}
\usepackage{amsmath}
\usepackage{amssymb}
\usepackage{marvosym} % additional math symbols
\usepackage{hyperref}
\usepackage[table, dvipsnames]{xcolor}
\definecolor{lightBlue}{RGB}{204, 229, 255}
\definecolor{hyperColor}{RGB}{5, 99, 183}
\usepackage{blindtext}
\usepackage{titlesec}
\hypersetup{
    colorlinks=true,
    linkcolor={hyperColor},
    citecolor={blue!50!black},
    urlcolor={hyperColor}
}
\usepackage{tocloft} % Control table of contents, figures, etc
\renewcommand{\cftsecleader}{\cftdotfill{\cftdotsep}}
\providecolors{dvipsnames*}
\usepackage{wrapfig}
\usepackage{graphicx}
\usepackage{amsfonts} %TeX fonts from the American Mathematical Society
\usepackage[usenames,dvipsnames]{color}
\usepackage[letterpaper, margin=1in]{geometry}
\usepackage{stmaryrd}
\usepackage{booktabs}
\usepackage{float}
\usepackage{floatrow}
%\usepackage{parskip} % normal paragraph structure



\usepackage{biblatex}
\usepackage[english]{babel}
\usepackage[square,numbers]{natbib}
\bibliographystyle{abbrvnat}






\author{Alexander, Yonatan\\
\texttt{SN: 251097927}
\and
Aziz, Mohammad Yasin\\
\texttt{SN: .....}
\and
Kien Tang Tran, Michael\\
\texttt{SN: 250735158}
}
\title{Therapy Support App}
% Document
\begin{document}

    TODO:

    \begin{enumerate}
        \item Business requirements related to problem's root cause
        \item Examples:
        \begin{enumerate}
            \item Accurate monthly sales forecast which will help us increase our sales
            \item need to generate monthly report that indicates sales
            \item notifications to account executives when a customer opens a problem ticket
        \end{enumerate}
        \item Varies based on process model
        \item Expected:
        \begin{enumerate}
            \item Shippable product
            \item some documentation
            \item lessons learned
            \item Why we chose one thing and not another
            \item If we switched from one platform to another (if we switched)
        \end{enumerate}
        \item stakeholders:
        \begin{enumerate}
            \item identify asap
            \item anyone who has any relation to the project
            \item external - anyone who has an influence\; including people who write certain libraries and so on  (include instructor and TA as external stakeholders)
            \item internal - programmers, maintainable, customers, users, etc [double check this point's list]
        \end{enumerate}
        \item success/acceptance criteria:
        \begin{enumerate}
            \item measurable terms of the project's outcome for the end user, customer, stakeholders, etc.
            \item cost
            \item timeline
            \item Business requirements
            \item scope
            \item acceptance criteria: (can be only a few)
            \begin{enumerate}
                \item conditions the software must fulfill
            \end{enumerate}
        \end{enumerate}
    \end{enumerate}



    \maketitle
%    Cover sheet {Project title, Group members' names and SID}

    \pagebreak

    \tableofcontents

    \pagebreak


    \section{Problem Definition}\label{sec:problem-definition}
    Therapists and patients encounter many challenges during the therapeutic process.
    There are a number of small yet persistent problems for therapists revolve around their patients' self-management between sessions.
    This software's intention is to help patients in doing so.
    Among the many problems, the following stand out (in the order of important:)
    \begin{enumerate}
        \item mood tracking between sessions
        \begin{enumerate}
            \item numerical (including descriptions of the numeric values, to increase clarity)
            \item statistical (based on numerical inputs)
            \item descriptive (``gloomy``, ``sad``, etc)
        \end{enumerate}
        \item Medication management
        \begin{itemize}
            \item Inventory
            \item Persistence (taking medications as required)
        \end{itemize}
        \item Symptoms
        \begin{itemize}
            \item Medical (unrelated to the patients diagnosed )
        \end{itemize}
    \end{enumerate}


    \section{Project Objective}\label{sec:project-objective}

    \subsection{General objective guidelines}\label{subsec:general-objective-guidelines}
    Patients should be able to generate a report about their mood and medication for their therapy session in one to three clicks between each therapy session.

    Therapists should be able to see their patient's mood and medication reports in one to two clicks between each therapy session.

    \begin{enumerate}
        \item Learning the process of project development through application of knowledge acquired during the Computer Science 4471A Course.
        \item The creation of an intuitive and user friendly self-management mobile app (Android and/or i/OS).
    \end{enumerate}


    \section{Stakeholders List}\label{sec:stakeholders-list}
    \begin{itemize}
        \item Patients - Internal stakeholder
        \item Therapists - External stakeholder
    \end{itemize}

    \subsection{Internal Stakeholders}\label{subsec:internal-stakeholders}

    \item external - anyone who has an influence\; including people who write certain libraries and so on  (include instructor and TA as external stakeholders)
    \item internal - programmers, maintainable, customers, users, etc [double check this point's list]

    \begin{enumerate}
        \item Patients (main users)
        \item Therapists (users by proxy, potential users if a therapist-side interface is implemented)
    \end{enumerate}

    \subsection{External Stakeholders}\label{subsec:external-stakeholders}
    \begin{enumerate}
        \item Legislation and public policy (for example security of medical information)
        \item Governing bodies (such as the CRPO - The College of Registered Psychotherapists of Ontario)
        \item Instructor and TA
    \end{enumerate}


    \section{Success/Acceptance Criteria for each Stakeholder}\label{sec:success/acceptance-criteria-for-each-stakeholder}

    \subsection{Patients}\label{subsec:patients}
    \begin{itemize}
        \item As a user, I want to be able to store my mood on a particular day.
        \item As a user, I want to be able to rate my mood on a particular day on a scale of 1 to 10.
        \item As a user, I want to be able to store my medication details such as the medication, dosage, date, and doctor.
        \item As a user, I want to be able to store when I take my medication.
        \item As a user, I want to be able to have an organized collection of mood and medication reports for my therapy visit.
        \item As a user, I want to be able to create a statistical report about my ratings on moods.

    \end{itemize}

    \subsection{Therapist}\label{subsec:therapist}
    \begin{itemize}
        \item As a user, I want to be able to look at my patient's reports in an organized manner.
        \item As a user, I want to be able to see the medication list stored on the application.

    \end{itemize}


    \section{Use case diagram(s)}\label{sec:use-case-diagram(s)}


    \section{Selected Use case Descriptions {only two descriptions}}\label{sec:selected-use-case-descriptions}


    \section{Sequence diagram(s) {for the selected use case for descriptions}}\label{sec:sequence-diagram(s)}


    \section{System Architecture}\label{sec:system-architecture}

    \begin{enumerate}
        \item \textbf{Repository architecture}: For medication database
        \item \textbf{Event driven architecture}: for reminders
        \item \textbf{Pipe and filters architecture}: for statistical analysis and graphs
        \item \textbf{Component based architecture}: for the whole system, especially for the user interface.
    \end{enumerate}


    \section{Detailed Class diagram(s)}\label{sec:detailed-class-diagram(s)}


    \section{State-machine diagram {for the whole system, if possible}}\label{sec:state-machine-diagram}


    \section{ER – Diagram (Data modelling)}\label{sec:er--diagram-(data-modelling)}


    \section{GitHub Link}\label{sec:github-link}


    \section{Conclusion}\label{sec:conclusion}


    \section{References}\label{sec:reference}

    \begin{enumerate}
        \item \textbf{MyTherapy (app):} \qq{Your personal pill reminder and medication tracker app}\cite{MyTherapy}
        \item \textbf{Bipolar UK’s Mood Tracker (app):} \qq{Our new Mood Tracker app can make it much easier to record your daily mood, medications, emotions and how much sleep you’ve had}\cite{BiPolUK}
    \end{enumerate}

    \bibliography{main}

    \printbibliography[heading=subbibintoc]

    \appendix{Project WBS}

    \appendix{Task Assignment Matrix}

    \begin{figure}
        \label{fig:Gant Chart}
        %GANT CHART
    \end{figure}

    \appendix{Sample of commits on the selected version control system}
    \appendix{things we want to do}
    \begin{enumerate}
        \item docker
        \item gant chart
        \item Reminder system
        \item Privacy
        \item Minutes
    \end{enumerate}

\end{document}
