%! Author = Jonathan
%! Date = 2022-09-26

% Preamble
\documentclass[11pt]{article}

% Packages
\usepackage{amsmath}
\author{Alexander, Yonatan\\
\texttt{SN: 251097927}
\and
Aziz, Mohammad Yasin\\
\texttt{SN: .....}
\and
Kien Tang Tran, Michael\\
\texttt{SN: .....}
}
\title{Therapy Support App}
% Document
\begin{document}

    TODO:

    \begin{enumerate}
        \item Business requirements related to problem's root cause
        \item Examples:
        \begin{enumerate}
            \item Accurate monthly sales forecast which will help us increase our sales
            \item need to generate monthly report that indicates sales
            \item notifications to account executives when a customer opens a problem ticket
        \end{enumerate}
        \item Varies based on process model
        \item Expected:
        \begin{enumerate}
            \item Shippable product
            \item some documentation
            \item lessons learned
            \item Why we chose one thing and not another
            \item If we switched from one platform to another (if we switched)
        \end{enumerate}
        \item stakeholders:
        \begin{enumerate}
            \item identify asap
            \item anyone who has any relation to the project
            \item external - anyone who has an influence\; including people who write certain libraries and so on  (include instructor and TA as external stakeholders)
            \item internal - programmers, maintainable, customers, users, etc [double check this point's list]
        \end{enumerate}
        \item success/acceptance criteria:
        \begin{enumerate}
            \item measurable terms of the project's outcome for the end user, customer, stakeholders, etc.
            \item cost
            \item timeline
            \item Business requirements
            \item scope
            \item acceptance criteria: (can be only a few)
            \begin{enumerate}
                \item conditions the software must fulfill
            \end{enumerate}
        \end{enumerate}
    \end{enumerate}



    \maketitle
    Cover sheet {Project title, Group members' names and SID}


    \chapter{Problem Definition}\label{ch:problem-definition}
    Therapists and patients encounter many challenges during the therapeutic process.
    There are a number of small yet persistent problems for therapists revolve around their patients' self-management between sessions.
    This software's intention is to help patients in doing so.
    Among the many problems, the following stand out (in the order of important:)
    \begin{enumerate}
        \item mood tracking between sessions
        \begin{enumerate}
            \item numerical (including descriptions of the numeric values, to increase clarity)
            \item statistical (based on numerical inputs)
            \item descriptive (``gloomy``, ``sad``, etc)
        \end{enumerate}
        \item Medication management
        \begin{itemize}
            \item Inventory
            \item Persistence (taking medications as required)
        \end{itemize}
        \item Symptoms
        \begin{itemize}
            \item Medical (unrelated to the patients diagnosed )
        \end{itemize}
    \end{enumerate}


    \chapter{Project Objective}\label{ch:project-objective}


    \chapter{Stakeholders List}\label{ch:stakeholders-list}


    \chapter{Success/Acceptance Criteria for each Stakeholder}


    \chapter{Use case diagram(s)}


    \chapter{Selected Use case Descriptions {only two descriptions}}


    \chapter{Sequence diagram(s) {for the selected use case for descriptions}}


    \chapter{System Architecture}


    \chapter{Detailed Class diagram(s)}


    \chapter{State-machine diagram {for the whole system, if possible}}


    \chapter{ER – Diagram (Data modelling)}


    \chapter{GitHub link to your project source code}


    \chapter{Conclusion (lesson learned)}


    \chapter{Reference (if any)}


    \chapter{Project WBS (as an appendix of your report)}


    \appendix{Task Assignment Matrix}

    \appendix{Sample of commits on the selected version control system}


    \bibliography{main}
    \bibliographystyle{plain}
\end{document}
